% !TEX program = pdflatex
\documentclass[a4]{article}


\usepackage[utf8]{inputenc}
\usepackage[british]{babel} % For the date to be written in a non-barbaric way

\usepackage{hyperref} % You need to load hyperref before knowledge
\usepackage{cleveref} % You do NOT need to load cleverref, but I highly recommand it!
\usepackage[xcolor, notion, quotation]{knowledge}
% To leave the "writting mode" and hide the anchor point, load knowledge with the "electronic" option.


\knowledge{url={https://ctan.org/pkg/knowledge}, typewriter}
 | knowledge@package

\knowledge{url={http://mirrors.ctan.org/macros/latex/contrib/knowledge/knowledge.pdf}}
 | documentation

\knowledge{url={https://ctan.org/pkg/hyperref?lang=en}, typewriter}
 | hyperref

\knowledge{text=\LaTeX}
 | latex

\knowledge{notion}
 | Knowledges@concept
 | knowledge@concept
 | knowledges@concept

\knowledge{notion}
 | Directives
 | directives
 | directive

\knowledge{notion, typewriter}
 | notion
 | notions

\knowledge{url={https://en.wikipedia.org/wiki/Tomato}, color=purple, boldface}
 | tomato
 | tomatoes
 | Tomato
 | Tomatoes

\knowledge{notion}
 | introduce
 | introduced
 | introduction

\knowledge{notion}
 | anchor points

\knowledge{notion}
 | composition mode

\knowledge{notion}
 | paper mode

\knowledge{notion}
 | electronic mode

\knowledge{notion}
 | scopes
 % notions for knowledge
\knowledgenewrobustcmd{\foo}{
    \cmdkl{f}
}

\knowledgenewrobustcmd{\foobar}[1]{
    \cmdkl{[} #1 \cmdkl{]}
} % (math) commands defined using knowledge


\title{Knowledge: minimal(ish) example}
\author{Rémi Morvan}
\date{\today}


\begin{document}

\maketitle


\section{Introduction}

This document is an example of {\LaTeX} document written using "knowledge", a package
written by "Thomas Colcombet", which allows you to \emph{easily} add internal (and external)
hyperlinks in your documents.

\AP Say that at some point of your document, you want to define "foo".
For instance, say that the ""foo"" of an integer $x$ is
defined as the smallest integer $y$ such that $y \geq x$, and is denoted
by $\intro*\foo(x)$.

\AP In this paragraph, you want to talk about "kittens". Unfortunately, part of your audience
might not know what "kittens" are, so you need to define the notion. A ""kitten"" is some
kind of cat, but slightly smaller and cuter. Note that ``"kittens"'' and ``"kitten"''
refer to the same notion (this is declared in the file \texttt{notions.tex}).


\section{My theorem}

In this section, you are finaly ready to prove your incredible theorem:
for every integer, the "foo" of $x$ is $x$ itself. If you wish, you can remind your reader
where "foo" was defined: it was in \kCref{foo}, Page \kpageref{foo}.

\section{A note on anchor points}

\AP On the left-hand side on the document, you can notice small red corners, called ""anchor points"".
Without them, the default behaviour of "knowledge" (inherited from \texttt{hyperref}) would be to
send you, for example when clicking on ``"foo"'' or ``"kitten"'', at the beginning of the section in which ``"foo"'' and
``kitten'' were defined. This is not precise enough, so you can add "anchor points" using the command
\texttt{\textbackslash AP}: thanks to those, ``"foo"'' and ``"kitten"'' are references to two different paragraphs of
the document

It is usefull to see your "anchor points" when writting a document, but you don't want to display them when compiling the final version of your document. To desactivate anchor points (and hide warning), you can add the option
\texttt{electronic} when loading "knowledge".


\section{Math commands}

You can define a ``partial link'' in some math command.
For example, say that you want a command \texttt{\textbackslash foobar} that takes an argument and displays this
argument between brackets: $\foobar{\phi}$.
You might want the argument $\phi$ not to link to the definition of \texttt{\textbackslash foobar}: you can easily
to that, by defining \AP \texttt{\textbackslash foobar} (producing $\intro*\foobar{x}$) as 
\begin{verbatim}
    \knowledgenewrobustcmd{\foobar}[1]{
        \cmdkl{[} #1 \cmdkl{]}
    }
\end{verbatim}
Only what is written inside \texttt{\textbackslash cmdkl} will be linked to the definition of the notion. 


\end{document}