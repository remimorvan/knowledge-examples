\documentclass{article}
\usepackage[utf8]{inputenc}
\usepackage[T1]{fontenc}
% If the output is a PDF (avoids some display problems, in particular in arXiv).
\pdfoutput = 1

%
% This is the file 
%
%   knowledge-example.tex
%
% and is part of the LaTeX package _knowledge_.
% It can be used as a starting point for creating a document using the _knowledge_ package.
%
% The documentation of the _knowledge_ package can be found:
% - in your tex distribution, using the command
%        texdoc knowledge
% - or at the address
%.       https://ctan.org/pkg/knowledge
%
%
% Loading other packages before knowledge activates features.
% The most common use of knowledge makes use of hyperref and xcolor:
\usepackage[breaklinks,hidelinks]{hyperref} 
\usepackage{xcolor} 
%
% The package 'knowledge' has  now to be loaded.
% The options 
%      'paper', 'electronic' or 'composition' (default)
% can be used. These activates different rendering styles:
% - 'paper' produce a paper to be printed:
%   text in black and white
% - 'electronic' highlights links using colors:
%   for being read on an electronic device
% - 'composition' (or default) highlights missing knowledges as
%   well as agives other pieces of information. It should always
%   be used but when the paper is ready.
%
\usepackage{knowledge} % default
%\usepackage[scope,electronic]{knowledge} % final version to be read electronically
%\usepackage[scope,paper]{knowledge} % final version in black and white for printing
%

% The 'notion' configuration is commonly used for scientific papers.
\knowledgeconfigure{notion}

% The 'quotation' configuration is commonly used and triggers the "..." notation.
\knowledgeconfigure{quotation}

% It is convenient to provide a list of \knowledge in an external file.
\knowledge{url={https://ctan.org/pkg/knowledge}, typewriter}
 | knowledge@package

\knowledge{url={http://mirrors.ctan.org/macros/latex/contrib/knowledge/knowledge.pdf}}
 | documentation

\knowledge{url={https://ctan.org/pkg/hyperref?lang=en}, typewriter}
 | hyperref

\knowledge{text=\LaTeX}
 | latex

\knowledge{notion}
 | Knowledges@concept
 | knowledge@concept
 | knowledges@concept

\knowledge{notion}
 | Directives
 | directives
 | directive

\knowledge{notion, typewriter}
 | notion
 | notions

\knowledge{url={https://en.wikipedia.org/wiki/Tomato}, color=purple, boldface}
 | tomato
 | tomatoes
 | Tomato
 | Tomatoes

\knowledge{notion}
 | introduce
 | introduced
 | introduction

\knowledge{notion}
 | anchor points

\knowledge{notion}
 | composition mode

\knowledge{notion}
 | paper mode

\knowledge{notion}
 | electronic mode

\knowledge{notion}
 | scopes



\title{Minimal example for the "knowledge" package}
\date{}
\author{}


\begin{document}

\maketitle


\begin{abstract}
  This document provides an elementary example of the "knowledge package" for "latex" and can be used as a starting point for creating one's own document.
\end{abstract}


\section{Introduction}
\label{section:introduction}

The package \AP""knowledge@@package"" is a package for "latex" that helps 
associating information to terms. This can be used for:
\begin{itemize}
	\item managing external urls, for instance separating the file containing  
    	the addresses from their use,
	\item managing internal references's such as linking every use of a concept 
    	to the place of its introduction (in particular avoiding the use of labels),
	\item managing the index in a centralized way,
	\item replacing some macros for configuring the display.
\end{itemize}

Primarily, the goal of the "knowledge@@package" is for the production of 
scientific documents (the longer, the more interesting, such as a thesis or a 
book) in order to improve their readability on electronic devices. Ultimately, 
the goal is to produce documents that are more semantic-aware. Some 
capabilities (link handling the index) are not related to this purpose in 
particular.

\section{Example running}
\label{section:example}

Try compiling this document (two compilation phases to have proper links) using 
"pdflatex", and see how some notions are hyperlinked to their introduction 
point (some viewers make it more obvious than others by displaying a preview of 
the target of a link inside a document; since there is only one page in this 
example, this may be worth zooming in this case).
When the "paper mode" is not active, links are clearly identified in blue. Try 
then compiling it in \AP""paper mode"" (an option of the "knowledge package"); 
it now looks like a regular paper (but the links are still there).
In \AP""electronic mode"", the links are still colored,
but some other hints disappear like "anchor points".


\knowledgeconfigure{quotation=false}% temporarily disables the " notation
	\AP While writing a document, the two syntaxes \verb|"text"| and
	\verb|""text""| should be used each time some important concept is used or 
	introduced respectively in the paper. For instance:
	\begin{quote}
	\verb|Define a ""group"" to be a "monoid" such that...|
	\end{quote}
	The `@' symbol allows some flexibility by having a displayed text different 
	from the target:
	\begin{quote}
	\verb|"This kind of algebras@solvable groups"...|
	\end{quote}
\knowledgeconfigure{quotation}

\AP These concepts are referred to as ""knowledges"". "Knowledges" are to be defined (in general in the preamble of the document) using a command of the form:

\begin{verbatim}
  \knowledge{directives}
    | name
    | synonym1
    | synonym2
    | ...
\end{verbatim}

However, using undefined "knowledges" will not cause compile time errors, 
but be displayed as in the following `"unknown knowledge"' (i.e. in 
non-"paper mode" in brown, but in regular black in "paper mode" or 
"electronic mode").
The \AP""diagnose file"" (that ends with a \texttt{.diagnose} extension) 
contains detailed information about these warnings, and should be read 
often when finalizing the document.

Small red corners called \AP""anchor points"" are visible in the margin, and 
represent the precise location in a pdf document where an internal link jumps. 
These are introduced
using the \texttt{\detokenize{\AP}}command, commonly at the beginning of each 
paragraph that introduce some concept, or even in the middle of the paragraph. 
The "anchor points" become invisible in "paper@paper mode" or "electronic mode".

\end{document}